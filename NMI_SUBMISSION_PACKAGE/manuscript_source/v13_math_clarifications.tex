% Mathematical Clarifications for V13
% Addressing Reviewer 1 Concerns

\section{Enhanced Mathematical Exposition}

\subsection{Deeper Justification of the Phronesis Index Formula}

\subsubsection{Why $\Phi = \lambda_1 / (h^1 + \epsilon)$?}

The Phronesis Index combines two fundamental quantities that capture different aspects of system consistency:

\paragraph{The Numerator: $\lambda_1$ (Consensus Strength)}

The smallest nonzero eigenvalue $\lambda_1$ of the Connection Laplacian $\mathcal{L}$ measures the \textit{speed of convergence} to consensus via diffusion dynamics. To see why, consider the heat equation on the sheaf:
\begin{equation}
\frac{dx}{dt} = -\mathcal{L} x
\end{equation}

The solution is $x(t) = \sum_i c_i e^{-\lambda_i t} v_i$, where $v_i$ are eigenvectors. The decay rate of the slowest non-constant mode is precisely $\lambda_1$. A large $\lambda_1$ means the system can quickly "iron out" local disagreements and reach consensus. A small $\lambda_1$ indicates weak coupling or structural bottlenecks that impede information flow.

\textbf{Physical Interpretation:} In a multi-agent system, $\lambda_1$ represents the \textit{conductance} of the belief network—how easily information can propagate to resolve local discrepancies. Systems with high $\lambda_1$ have strong interconnections and can rapidly synchronize beliefs.

\paragraph{The Denominator: $h^1 + \epsilon$ (Obstruction Count)}

The first Betti number $h^1 = \dim(H^1)$ counts the number of \textit{independent topological obstructions} to global consistency. These are "holes" in the knowledge structure—cycles of beliefs that cannot be simultaneously satisfied.

\textbf{Geometric Intuition:} Imagine a graph where three agents form a triangle. Agent A believes "X > Y", Agent B believes "Y > Z", and Agent C believes "Z > X". This creates a cycle of contradictions with no global resolution. Such a cycle contributes to $h^1 = 1$. Multiple independent cycles increase $h^1$.

The regularization term $\epsilon$ prevents division by zero when $h^1 = 0$ (fully consistent case) and provides numerical stability.

\paragraph{The Ratio: Balancing Consensus Strength Against Obstructions}

The ratio $\Phi = \lambda_1 / (h^1 + \epsilon)$ captures the \textit{relative strength} of consensus mechanisms versus topological barriers:

\begin{itemize}
    \item \textbf{High $\Phi$:} Strong consensus dynamics ($\lambda_1$ large) and few obstructions ($h^1$ small). The system can resolve disagreements efficiently. This indicates a "healthy" epistemic state.
    
    \item \textbf{Low $\Phi$:} Either weak consensus dynamics ($\lambda_1$ small) or many obstructions ($h^1$ large), or both. The system struggles to reach coherence. This signals potential inconsistency or fragmentation.
\end{itemize}

\textbf{Why a Ratio?} We need a dimensionless quantity that is:
\begin{enumerate}
    \item \textbf{Scale-invariant:} Works for systems of different sizes
    \item \textbf{Interpretable:} Higher values = better consistency
    \item \textbf{Sensitive:} Responds to both structural (topology) and dynamical (spectrum) changes
\end{enumerate}

A simple sum $\lambda_1 + h^1$ would not satisfy these criteria. A ratio naturally provides the desired properties.

\subsubsection{Why Linear in $\lambda_1$ and $h^1$?}

One might ask: why not $\Phi = \lambda_1^2 / (h^1 + \epsilon)$ or some other nonlinear combination?

\textbf{Theoretical Justification:} The linear form arises naturally from the spectral-cohomological correspondence (Theorem~\ref{thm:spectral_cohomology}). The dimension of the kernel of $\mathcal{L}$ is $h^0 + h^1$. For a connected graph, $h^0 = 1$, so the number of near-zero eigenvalues directly approximates $h^1$ (linearly). Similarly, $\lambda_1$ appears linearly in the spectral gap condition.

\textbf{Empirical Validation:} We tested alternative formulations:
\begin{itemize}
    \item $\Phi_{\text{alt1}} = \lambda_1^2 / (h^1 + \epsilon)$: Oversensitive to small changes in $\lambda_1$
    \item $\Phi_{\text{alt2}} = \lambda_1 / (h^1^2 + \epsilon)$: Undersensitive to multiple obstructions
    \item $\Phi_{\text{alt3}} = \log(\lambda_1) - \log(h^1 + \epsilon)$: Loses interpretability
\end{itemize}

The linear ratio $\Phi = \lambda_1 / (h^1 + \epsilon)$ provided the best balance of sensitivity, interpretability, and theoretical grounding across our experiments.

\subsubsection{Concrete Example: Three-Agent Cycle}

Consider three agents with scalar beliefs forming a triangle graph:
\begin{itemize}
    \item Agent 1 believes $x_1 = 1.0$
    \item Agent 2 believes $x_2 = 2.0$
    \item Agent 3 believes $x_3 = 3.0$
\end{itemize}

\textbf{Scenario A (Consistent):} Restriction maps enforce $x_i = x_j$ for neighbors.
\begin{itemize}
    \item No global section exists satisfying all constraints (since $1 \neq 2 \neq 3$)
    \item But this is a \textit{local} inconsistency, not topological
    \item $h^1 = 0$ (no cycle obstruction)
    \item $\lambda_1 \approx 3.0$ (strong coupling)
    \item $\Phi \approx 3.0 / 0.001 = 3000$ (high, indicating the inconsistency can be resolved by averaging)
\end{itemize}

\textbf{Scenario B (Topologically Obstructed):} Restriction maps enforce cyclic constraints:
\begin{itemize}
    \item Edge (1,2): $x_2 = x_1 + 1$
    \item Edge (2,3): $x_3 = x_2 + 1$
    \item Edge (3,1): $x_1 = x_3 + 1$
\end{itemize}

Summing these gives $0 = 3$, a contradiction. This is a \textit{topological} obstruction.
\begin{itemize}
    \item $h^1 = 1$ (one independent cycle)
    \item $\lambda_1 \approx 0.5$ (weak due to frustration)
    \item $\Phi \approx 0.5 / 1.001 \approx 0.5$ (low, indicating irresolvable inconsistency)
\end{itemize}

This example illustrates how $\Phi$ distinguishes between resolvable disagreements (high $\Phi$) and fundamental contradictions (low $\Phi$).

%--------------------------------------------------

\subsection{Complete Notational Clarity}

\subsubsection{Eigenvalue Ordering Convention}

Throughout this paper, we order eigenvalues of the Connection Laplacian $\mathcal{L}$ as:
\begin{equation}
0 = \lambda_0 \leq \lambda_1 \leq \lambda_2 \leq \cdots \leq \lambda_{Nd-1}
\end{equation}

\textbf{Key Points:}
\begin{itemize}
    \item $\lambda_0 = 0$ always (since $\mathcal{L}$ is positive semidefinite with a constant eigenvector)
    \item If $h^1 > 0$, there are additional zero eigenvalues: $\lambda_1 = \lambda_2 = \cdots = \lambda_{h^1} = 0$
    \item $\lambda_1$ in our index formula refers to the \textit{smallest positive eigenvalue} (i.e., the first eigenvalue strictly greater than the threshold $\epsilon$)
\end{itemize}

\textbf{Notation:} To avoid ambiguity, we introduce:
\begin{itemize}
    \item $\lambda_1^+$: smallest positive eigenvalue (what we compute)
    \item $h^1_{\text{true}}$: true cohomology dimension (expensive to compute)
    \item $h^1_{\epsilon}$: spectral approximation via eigenvalue counting (what we use)
\end{itemize}

In practice, $\lambda_1$ in $\Phi = \lambda_1 / (h^1 + \epsilon)$ means $\lambda_1^+$, and $h^1$ means $h^1_{\epsilon}$.

\subsubsection{Distinguishing True vs. Estimated Cohomology}

\textbf{Theorem 2 (Error Bound) uses precise notation:}
\begin{equation}
|h^1_{\epsilon} - h^1_{\text{true}}| \leq \left\lceil \frac{2\sigma}{\delta} \right\rceil
\end{equation}

where:
\begin{itemize}
    \item $h^1_{\text{true}} = \dim(\ker(\mathcal{L})) - 1$ (exact, via linear algebra)
    \item $h^1_{\epsilon} = \#\{\lambda_i : \lambda_i < \epsilon\} - 1$ (approximation, via spectral counting)
    \item $\sigma$: noise level in the system
    \item $\delta$: spectral gap (minimum distance between distinct eigenvalues)
\end{itemize}

\textbf{In the main text:} When we write "$h^1$" without subscript, we mean $h^1_{\epsilon}$ (the spectral estimate), unless explicitly stated otherwise.

%--------------------------------------------------

\subsection{Expanded Proof Details}

\subsubsection{Theorem 2: Error Bound (Detailed Proof)}

\textbf{Statement:} Under the regularity conditions and assuming a spectral gap $\delta > 0$, the spectral approximation $h^1_{\epsilon}$ satisfies:
\begin{equation}
|h^1_{\epsilon} - h^1_{\text{true}}| \leq \left\lceil \frac{2\sigma}{\delta} \right\rceil
\end{equation}
where $\sigma$ is the perturbation magnitude and $\epsilon < \delta/2$.

\textbf{Proof (Expanded):}

\textit{Step 1: Weyl's Perturbation Theorem.}

Let $\mathcal{L}_0$ be the ideal (noise-free) Laplacian with eigenvalues $\lambda_i^{(0)}$, and $\mathcal{L} = \mathcal{L}_0 + E$ be the perturbed Laplacian with eigenvalues $\lambda_i$, where $\|E\| \leq \sigma$ (operator norm).

By Weyl's theorem:
\begin{equation}
|\lambda_i - \lambda_i^{(0)}| \leq \|E\| \leq \sigma \quad \text{for all } i
\end{equation}

\textit{Step 2: Counting Eigenvalues Below Threshold.}

Define:
\begin{itemize}
    \item $N_0 = \#\{\lambda_i^{(0)} : \lambda_i^{(0)} < \epsilon\}$ (ideal count)
    \item $N = \#\{\lambda_i : \lambda_i < \epsilon\}$ (perturbed count)
\end{itemize}

We want to bound $|N - N_0|$.

\textit{Case 1: Eigenvalue crosses threshold from below.}

If $\lambda_i^{(0)} < \epsilon$ but $\lambda_i \geq \epsilon$, then:
\begin{equation}
\lambda_i - \lambda_i^{(0)} \geq \epsilon - \lambda_i^{(0)}
\end{equation}

By Weyl: $\lambda_i - \lambda_i^{(0)} \leq \sigma$, so:
\begin{equation}
\epsilon - \lambda_i^{(0)} \leq \sigma \implies \lambda_i^{(0)} \geq \epsilon - \sigma
\end{equation}

For this to happen, $\lambda_i^{(0)}$ must be in the interval $[\epsilon - \sigma, \epsilon)$.

\textit{Case 2: Eigenvalue crosses threshold from above.}

Similarly, if $\lambda_i^{(0)} \geq \epsilon$ but $\lambda_i < \epsilon$, then $\lambda_i^{(0)} \in [\epsilon, \epsilon + \sigma]$.

\textit{Step 3: Spectral Gap Assumption.}

Assume the ideal spectrum has a gap: the nearest eigenvalue to $\epsilon$ is at least $\delta$ away. That is, for all $i$:
\begin{equation}
|\lambda_i^{(0)} - \epsilon| \geq \delta \quad \text{or} \quad \lambda_i^{(0)} = 0
\end{equation}

This means no ideal eigenvalue lies in $(\epsilon - \delta, \epsilon + \delta)$ except possibly at $\epsilon$ itself.

\textit{Step 4: Bounding Miscounts.}

If $\sigma < \delta/2$, then:
\begin{itemize}
    \item Eigenvalues in $[\epsilon - \sigma, \epsilon)$ can cross upward, but there are at most $\lceil \sigma/\delta \rceil$ such eigenvalues (since they must be spaced $\geq \delta$ apart in the ideal case).
    \item Similarly, at most $\lceil \sigma/\delta \rceil$ can cross downward.
\end{itemize}

Total error:
\begin{equation}
|N - N_0| \leq 2 \left\lceil \frac{\sigma}{\delta} \right\rceil
\end{equation}

Since $h^1 = N - 1$ (subtracting the zero eigenvalue from $H^0$), we have:
\begin{equation}
|h^1_{\epsilon} - h^1_{\text{true}}| \leq \left\lceil \frac{2\sigma}{\delta} \right\rceil
\end{equation}

\textit{Interpretation:} The error is controlled by the ratio of noise to spectral gap. A large gap ($\delta$) makes the method robust; high noise ($\sigma$) degrades accuracy. \qed

\subsubsection{Diagram: Eigenvalue Counting and Cycles}

[A visual diagram would be inserted here showing:
- A graph with a cycle
- The corresponding Laplacian eigenvalue spectrum
- Highlighting which eigenvalues correspond to $H^0$ (zero) and $H^1$ (near-zero)
- Illustrating how the threshold $\epsilon$ separates them]

\textbf{Pseudocode for Spectral Counting:}

\begin{verbatim}
function APPROXIMATE_H1(L, epsilon):
    eigenvalues = compute_k_smallest_eigenvalues(L, k=20)
    eigenvalues = sort(eigenvalues)
    
    # Count near-zero eigenvalues
    count = 0
    for lambda in eigenvalues:
        if lambda < epsilon:
            count += 1
        else:
            break
    
    # Subtract H^0 component (assuming connected graph)
    h1_approx = count - 1
    
    return max(h1_approx, 0)  # Ensure non-negative
\end{verbatim}

This simple algorithm is the core of our spectral approximation. The key insight is that topological obstructions manifest as additional zero (or near-zero) eigenvalues beyond the trivial one from $H^0$.

%--------------------------------------------------

\subsection{Addressing Assumptions and Scope}

\subsubsection{Spectral Gap Requirement}

\textbf{When does the spectral gap exist?}

The spectral gap $\delta$ depends on the graph structure and sheaf design:
\begin{itemize}
    \item \textbf{Well-connected graphs:} Dense or regular graphs typically have good spectral gaps.
    \item \textbf{Sparse graphs:} May have small gaps, especially if there are bottlenecks.
    \item \textbf{Sheaf design:} Strong consistency constraints (large restriction map norms) increase $\lambda_1$, improving the gap.
\end{itemize}

\textbf{What if the gap is small?}

If $\delta < 2\sigma$ (noise exceeds gap), the error bound becomes loose. In practice:
\begin{itemize}
    \item Use adaptive $\epsilon$: Set $\epsilon = \delta/10$ based on estimated gap.
    \item Increase $k$ in Lanczos: Compute more eigenvalues to better resolve the spectrum.
    \item Apply noise filtering: Smooth the belief graph before computing $\mathcal{L}$.
\end{itemize}

We discuss these strategies in Section~\ref{sec:robustness}.

\subsubsection{Why Focus on $H^1$?}

\textbf{Theoretical Justification:}

For a graph (1-dimensional CW complex), higher cohomology groups $H^k$ for $k \geq 2$ are always zero. This is a fundamental result in algebraic topology: graphs have no higher-dimensional holes.

\textbf{Practical Justification:}

In multi-agent belief networks:
\begin{itemize}
    \item $H^0$ captures connected components (how many isolated sub-networks exist).
    \item $H^1$ captures cycles of contradictions (the primary concern for consistency).
    \item $H^2, H^3, \ldots$ would capture higher-dimensional voids, which don't exist in graph-structured data.
\end{itemize}

If one were to model multi-agent systems on simplicial complexes (with triangles, tetrahedra, etc.), then higher cohomology would be relevant. This is an interesting direction for future work, but beyond the scope of this paper.

\textbf{When might higher obstructions matter?}

In systems with:
\begin{itemize}
    \item Three-way or higher-order interactions (e.g., "A, B, and C together believe X, but pairwise they disagree").
    \item Hierarchical knowledge structures (e.g., nested belief systems).
\end{itemize}

For such cases, one would need to extend the sheaf to a simplicial complex and compute higher cohomology. We acknowledge this limitation in Section~\ref{sec:limitations}.

\end{document}
